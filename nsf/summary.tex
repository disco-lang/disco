\centerline{\Large DISCO: A PROGRAMMING ENVIRONMENT}
\vspace{1ex}
\centerline{\Large FOR DISCRETE MATHEMATICS EDUCATION}
\vspace{3ex}
\centerline{\large PIs: Harley Eades and Brent Yorgey}
\vspace{3ex}

\begin{comment}
  The proposal must contain a summary of the proposed activity suitable
  for publication, not more than one page in length. It should not be an
  abstract of the proposal, but rather a self-contained description of the
  activity that would result if the proposal were funded. The summary
  should be written in the third person and include a statement of
  objectives and methods to be employed. It must clearly address in
  separate statements (within the one-page summary): (1) the intellectual
  merit of the proposed activity; and (2) the broader impacts resulting
  from the proposed activity. (See Chapter III for further descriptive
  information on the NSF merit review criteria.) It should be informative
  to other persons working in the same or related fields and, insofar as
  possible, understandable to a scientifically or technically literate lay
  reader. Proposals that do not separately address both merit review
  criteria within the one page Project Summary will be returned without
  review.
\end{comment}

A course in \emph{discrete mathematics} is typically a core
requirement for undergraduates pursuing a degree in computer science,
and introduces mathematical structures and techniques of foundational
importance in the field.  However, many students struggle to see the
relevance of the course content or to connect it to any of their
previous knowledge.

\emph{Functional programming} is a programming paradigm organized
around functions rather than sequences of instructions.  It enables
working at high levels of abstraction as well as rapid prototyping and
refactoring, and is seeing increasing adoption in industry.  In order
to prepare computer science students to be effective and engaged in
the world they will enter after graduation, it's critical that they
are introduced to functional programming; however, many departments
struggle to fit it into their curriculum.

Combining discrete mathematics and the basics of functional
programming into a single course is a promising solution to both
problems.  The topics complement each other well; students are better
able to see the relevance of discrete mathematics concepts when they
can explore them in a computational way; and functional programming
can be introduced early in the curriculum.  However, doing this comes
with problems of its own. Most existing functional languages are not
designed with teaching in mind. Moreover, the use of a large, complex,
general-purpose language makes it hard to get buy-in from faculty who
may not have expertise in the language.

{\bf Objectives: } The objective of this project is to develop a new
programming language, web-based interactive environment, and
curriculum, specifically designed to support students in exploring
discrete mathematics and basic functional programming.

{\bf Methodology: } The project will apply proven programming language
technologies and design principles to develop a programming language
and interactive web-based environment tailored for discrete
mathematics education.

In parallel with development of the programming environment, the
project will also develop curricular modules covering standard
discrete math and functional programming topics making central use of
the programming environment.

These curricular modules will be piloted in a discrete mathematics
course at one of the PI institutions in the first year.  In subsequent
years, instructors at other institutions will be offered a small
stipend to beta-test a revised version of the course.  Beta-test
instructors will also assist with data collection by administering
surveys to their students.

{\bf Intellectual Merit:} Designing a programming environment
which is easy for students to learn and use and gives them genuine
insight into discrete mathematics concepts will be an intellectually
demanding challenge.  Although education, not computer science
research, is the focus of this proposal, it is easy to imagine the
educational design work leading to interesting technical challenges
and publishable results in the field of programming languages.

{\bf Broader Impact: } The broader impact of the proposed work will
include the eventual adoption of the new programming environment and
curricular materials at many schools around the country.  The
programming language and web-based environment will be made freely
available, and all libraries and source code developed by the project
team will be open source and freely available.

In addition, the project will serve as an opportunity for mentorship
and involvement of undergraduates in research.  Particular effort will
be made to incorporate students from underrepresented groups.

{\bf Key Words:} Discrete mathematics, functional programming, web


%%% Local Variables:
%%% mode: latex
%%% TeX-master: "proposal"
%%% End:
