%% -*- mode: LaTeX; compile-command: "./build.sh" -*-
\documentclass[xcolor=svgnames,12pt,aspectratio=169]{beamer}

\usepackage[all]{xy}
\usepackage{brent}
\usepackage{xspace}
\usepackage{fancyvrb}
\usepackage{ulem}

\usepackage{ucs}
\usepackage[utf8x]{inputenc}
\usepackage[backend=pgf,extension=pgf,input,outputdir=diagrams]{diagrams-latex}
\graphicspath{{images/}}

%%%%%%%%%%%%%%%%%%%%%%%%%%%%%%%%%%%%%%%%%%%%%%%%%%%%%%%%%%%%
%%%%%%%%%%%%%%%%%%%%%%%%%%%%%%%%%%%%%%%%%%%%%%%%%%%%%%%%%%%%

% Math typesetting

%% a bit more space for matrices
\setlength{\arraycolsep}{5pt}

\newcommand{\ty}[3]{{#1} \vdash {#2} : {#3}}
\newcommand{\nty}[3]{{#1} \nvdash {#2} : {#3}}

%%%%%%%%%%%%%%%%%%%%%%%%%%%%%%%%%%%%%%%%%%%%%%%%%%%%%%%%%%%%
%%%%%%%%%%%%%%%%%%%%%%%%%%%%%%%%%%%%%%%%%%%%%%%%%%%%%%%%%%%%

\newcommand{\etc}{\textit{etc.}}
\renewcommand{\eg}{\textit{e.g.}\xspace}
\renewcommand{\ie}{\textit{i.e.}\xspace}

\newcommand{\theschool}{IFIP WG 2.1}
% \newcommand{\thelocation}{Taiwan}
\newcommand{\thedate}{March 2019}

%%%%%%%%%%%%%%%%%%%%%%%%%%%%%%%%%%%%%%%%%%%%%%%%%%%%%%%%%%%%
%%%%%%%%%%%%%%%%%%%%%%%%%%%%%%%%%%%%%%%%%%%%%%%%%%%%%%%%%%%%

\setbeamertemplate{items}[circle]

\mode<presentation>
{
  \usetheme{default}                          % use a default (plain) theme

  \setbeamertemplate{navigation symbols}{}    % don't show navigation
                                              % buttons along the
                                              % bottom
  \setbeamerfont{normal text}{family=\sffamily}

  % XX remove this before giving actual talk!
  % \setbeamertemplate{footline}[frame number]
  % {%
  %   \begin{beamercolorbox}{section in head/foot}
  %     \vskip2pt
  %     \hfill \insertframenumber
  %     \vskip2pt
  %   \end{beamercolorbox}
  % }

  \AtBeginSection[]
  {
    \begin{frame}<beamer>
      \frametitle{}

      \begin{center}
%        \includegraphics[width=2in]{\sectionimg}
%        \bigskip

        {\Huge \insertsectionhead}
      \end{center}
    \end{frame}
  }
}

\defbeamertemplate*{title page}{customized}[1][]
{
  \vbox{}
  \vfill
  \begin{centering}
    \begin{beamercolorbox}[sep=8pt,center,#1]{title}
      \usebeamerfont{title}\inserttitle\par%
      \ifx\insertsubtitle\@@empty%
      \else%
        \vskip0.25em%
        {\usebeamerfont{subtitle}\usebeamercolor[fg]{subtitle}\insertsubtitle\par}%
      \fi%
    \end{beamercolorbox}%
    \vskip1em\par
    {\usebeamercolor[fg]{titlegraphic}\inserttitlegraphic\par}
    \vskip1em\par
    \begin{beamercolorbox}[sep=8pt,center,#1]{author}
      \usebeamerfont{author}\insertauthor
    \end{beamercolorbox}
    \begin{beamercolorbox}[sep=8pt,center,#1]{institute}
      \usebeamerfont{institute}\insertinstitute
    \end{beamercolorbox}
    \begin{beamercolorbox}[sep=8pt,center,#1]{date}
      \usebeamerfont{date}\insertdate
    \end{beamercolorbox}
  \end{centering}
  \vfill
}

\newenvironment{xframe}[1][]
  {\begin{frame}[fragile,environment=xframe,#1]}
  {\end{frame}}

% uncomment me to get 4 slides per page for printing
% \usepackage{pgfpages}
% \pgfpagesuselayout{4 on 1}[uspaper, border shrink=5mm]

\setbeameroption{show notes}

\renewcommand{\emph}{\textbf}

\title{Disco: A Functional Teaching Language for Discrete Mathematics}
\date{\theschool \\ \thedate}
\author{Brent Yorgey}
% \titlegraphic{\includegraphics[width=1in]{deriv-tree}}

%%%%%%%%%%%%%%%%%%%%%%%%%%%%%%%%%%%%%%%%%%%%%%%%%%%%%%%%%%%%

\begin{document}

\begin{xframe}{}
   \titlepage
\end{xframe}

\begin{xframe}{}
  XXX collaborators
\end{xframe}

\begin{xframe}{Context}
  \begin{itemize}
  \item<+-> I teach at a small undergraduate institution
  \item<+-> Want to introduce strongly typed FP to students early
  \item<+-> Can't use FP for intro programming course
  \item<+-> Can't add another required course
  \item<+-> Solution: FP in Discrete Math!
  \end{itemize}
\end{xframe}

\begin{xframe}{FP and Discrete Math}
  \begin{itemize}
  \item<+-> Already done in many places.
  \item<+-> \dots but using a general-purpose FP language (Haskell,
    OCaml, \dots)
  \item<+-> Seemed like a fun opportunity to design a new language.
  \end{itemize}
\end{xframe}

% \begin{xframe}{Goals}
%   \begin{itemize}
%   \item Share some interesting language design with you
%   \item Get some feedback and ideas
%   \item Find out if there's other related work I'm missing
%   \end{itemize}
% \end{xframe}

\section{Design principles}

\note{With this context in mind, let me tell you the core design
  principles of disco.}

\begin{xframe}
  \begin{center}
    {\Large \#1} \bigskip

    {\Large Elegant, strong static type system}
  \end{center}
\note{First, the language needs to have a well-designed, strong,
  static type system.  I want students to see the beauty and utility
  of thinking with types, and learn general principles that they can
  transfer to other languages.}

\end{xframe}

\begin{xframe}
  \begin{center}
    {\Large \#2} \bigskip

    {\Large Usable by beginning programmers}
  \end{center}
  \note{Some of the students taking the course will have taken an
    introductory programming course, but it shouldn't have programming
    as a prerequisite.}
\end{xframe}

\begin{xframe}
  \begin{center}
    {\Large \#3} \bigskip

    {\Large Syntax and semantics inspired by mathematical practice}
  \end{center}

  \note{As much as possible, I want to close the gap between the
    mathematics the students are learning in the class and the kinds
    of things they can talk about in the language.  I also want to
    make it attractive/possible for mathematics faculty to be able to
    pick it up quickly.}
\end{xframe}

\section{Types}

\note{OK, so let's talk about some interesting design choices
  underlying disco's type system. XXX polymorphism.  Let's start with
  thinking about base types.}

\begin{xframe}{Base types}
  \Huge
  \begin{overprint}
    \onslide<2> \[ \mathbb{N} \]
    \onslide<3> \[ \mathbb{N} \quad \mathbb{Z} \]
    \onslide<4> \[ \mathbb{N} \stackrel{\small ?}{\leq} \mathbb{Z} \]
  \end{overprint}
  \note{Any language for FP and discrete math obviously has to start
    with the natural numbers as a built-in base type!  And then we
    quickly want negative numbers too, so we add an integer type.  But
  then we immediately run head-on into the issue of subtyping. Do we
  want N to be a subtype of Z?}
\end{xframe}

\begin{xframe}{Base types}
  \Huge
    \[ \mathbb{N} \leq \mathbb{Z} \]
  \note{Based on principle \#3, I would argue that yes, we have to!
    Every student knows that natural numbers ARE integers.  It would
    be very jarring to be required to apply some sort of explicit
    conversion function.}
\end{xframe}

\begin{xframe}{Base types}
  \Huge
  \begin{overprint}
    \onslide<1> \[ \mathbb{R}? \]
    \onslide<2> \[ \not\mathbb{R} \]
  \end{overprint}
  \note{You might think the next base type we want is some sort of
    floating-point type. But we definitely don't.  Floating point
    numbers suck, and we don't need them for discrete math!}
\end{xframe}

\begin{xframe}{Base types}
  \Huge
  \[ \mathbb{N} \leq \mathbb{Z} \leq \mathbb{Q} \]
  \note{Instead, we make rational numbers built-in first-class
    citizens, so we can have division.  And then after a while of
    working with this, I had a sudden epiphany that there's a fourth
    type we want to add.  Does anyone know what I'm going to say?
    Maybe others have already thought about this?}
\end{xframe}

\begin{xframe}{Base types}
  \Large
  \[ \mathbb{F} = \left\{ \frac{a}{b} \mid a, b \in \mathbb{N}, b \neq 0 \right\} \]
\end{xframe}

\begin{xframe}{Base types}
  \begin{center}
  \begin{diagram}[width=100]
    dia = mconcat (zipWith moveTo (square 6 # rotateBy (-1/8)) baseTyDias)
      # connect' opts "N" "Z"
      # connect' opts "N" "F"
      # connect' opts "F" "Q"
      # connect' opts "Z" "Q"

    opts = with & gaps .~ local 1.3 & headLength .~ local 0.6

    baseTys = ["N", "Z", "Q", "F"]
    mkBaseTyDia ty = mconcat
      [ text ("$\\mathbb{" ++ ty ++ "}$")
      , square 1 # lw none
      ]
      # named ty
    baseTyDias = map mkBaseTyDia baseTys
  \end{diagram}
  \end{center}
  \note{Now we get this lovely subtyping lattice.  $\mathbb{N}$
    supports addition and multiplication.  Moving up and to the right
    corresponds to having additive inverses; up and left is
    multiplicative inverses. We can go back down with absolute value
    or with floor/ceiling.}
  % XXX if time, add more info to the picture
\end{xframe}

\begin{xframe}
  \begin{center}
    {\Huge Demo!}
  \end{center}
  \note{Do a quick demo.  Show basic arithmetic operations and
    inferring their types.  Show inferring the type of lists containing
    all naturals, negative, fraction, or both.  Show how this lets us
    e.g. be precise about the type of exponentiation, $2^5$, $(-2)^5$,
    $2^{-5}$, $(-2)^{-5}$.}
\end{xframe}

\begin{xframe}{Quiz}
  \begin{center}
    Q: What is the type of $\lambda x. x - 1$?
  \end{center}

  \onslide<2>
  A: it can be $\mathbb{Z} \to \mathbb{Z}$ or $\mathbb{Q} \to
  \mathbb{Q}$, but neither is more general than the other.
\end{xframe}

\begin{xframe}{Qualified types}
  \[ \lambda x. x - 1 : \forall \alpha. \text{sub } \alpha \Rightarrow
    \alpha \to \alpha \]

  \onslide<2-> \dots but we don't want students to see this!
  \note{Qualifiers are generated and used under the hood but always
    monomorphized away before showing the user.  Can show demo.}
\end{xframe}

\begin{xframe}
    \begin{overprint}
      \onslide<1>
      \begin{center}
        Qualified polymorphism + subtyping $\to$ yikes. \bigskip
      \end{center}
    \onslide<2->
    \begin{center}
      Qualified polymorphism + \textcolor{red}{structural} subtyping
      is just barely tractable!
    \end{center}
    \end{overprint}

    \onslide<3>
    \begin{center}
      Traytel et al. (2011???), extended to handle container
      subtyping.
    \end{center}
\end{xframe}

\section{Containers}

\begin{xframe}{B\"ohm Hierarchy}
  \[ \texttt{List} \leq \texttt{Bag} \leq \texttt{Set} \]
\end{xframe}

\begin{xframe}
  Gibbons $\Rightarrow$ ringad XXX comprehensions.
\end{xframe}

\begin{xframe}{Primitives}
  \begin{itemize}
  \item \texttt{list}, \texttt{bag}, \texttt{set}
  \item $\texttt{map} : (a \to b) \to C\ a \to C\ b$
  \item $\texttt{reduce} : (a \to a \to a) \to a \to C\ a \to a$
  \item $\texttt{mapReduce} : (a \to b) \to (b \to b \to b) \to b \to
    C\ a \to b$
  \item $\texttt{countBag} : \texttt{Bag}\ a \to \texttt{Set}\ (a
    \times \mathbb{N})$
  \item \texttt{union}, \texttt{intersection}, \texttt{difference}, $\dots$
  \end{itemize}
\end{xframe}

\section{Randomness}

\note{It would be nice to be able to have randomness.  But there are
  several constraints and I haven't yet found a satisfactory way to
  navigate the tradeoffs.}

\begin{xframe}{Constraints}
  \begin{itemize}
  \item<+-> Would like the types to reflect whether randomness is
    present.  Don't just want to give the whole language
    nondeterministic semantics.
  \item<+-> Don't want to make students use a monad!
  \item<+-> Did I mention the language is lazy?
  \end{itemize}
  \begin{center}
    \onslide<4>{Problem: how to control when sampling happens?}
  \end{center}
\end{xframe}

\section{Other fun stuff}

\begin{xframe}{Equirecursive types}
  \begin{overprint}

    \onslide<1>
\begin{verbatim}
type T = Unit + Nat * T * T

foldT : r -> (N -> r -> r -> r) -> T -> r
foldT x f (left  ())      = x
foldT x f (right (n,l,r)) = f n (foldT x f l) (foldT x f r)

sumT : T -> N
sumT = foldT 0 (\x l r. x + l + r)
\end{verbatim}

    \onslide<2>
\begin{verbatim}
type DoubleStream = N * N * DoubleStream

cons : N * DoubleStream -> DoubleStream
cons x = x
\end{verbatim}
  \end{overprint}
  \note{Ultimately this works because we require top-level type
    signatures on everything, and only top-level bindings can be
    recursive.}
\end{xframe}

\begin{xframe}{Arithmetic patterns}
\begin{verbatim}
h : N -> N
h(0)    = 1
h(2k+1) = h(k)
h(2k+2) = h(k+1) + h(k)
\end{verbatim}
\end{xframe}

\begin{xframe}{Ellipses a la Babbage}
\begin{verbatim}
Disco> [1 .. 10]
[1, 2, 3, 4, 5, 6, 7, 8, 9, 10]
Disco> [1, 3 .. 10]
[1, 3, 5, 7, 9]
Disco> [1, 3, 6 .. 100]
[1, 3, 6, 10, 15, 21, 28, 36, 45, 55, 66, 78, 91]
Disco> [1, 4, 9 .. 100]
[1, 4, 9, 16, 25, 36, 49, 64, 81, 100]
\end{verbatim}
\end{xframe}

\end{document}