% -*- compile-command: "rubber -d disco-tfpie23.tex" -*-

\documentclass[submission,copyright,creativecommons]{eptcs}
\providecommand{\event}{TFPIE 2023} % Name of the event you are submitting to

\usepackage{iftex}

\usepackage{underscore}         % Only needed if you use pdflatex.
\usepackage[T1]{fontenc}        % Recommended with pdflatex
\usepackage[utf8]{inputenc}
\usepackage{stmaryrd}
\usepackage{amssymb}

\newcommand{\N}{\mathbb{N}}

\DeclareUnicodeCharacter{27C5}{$\Lbag$}
\DeclareUnicodeCharacter{27C6}{$\Rbag$}
\DeclareUnicodeCharacter{2115}{$\N$}
\DeclareUnicodeCharacter{2192}{$\to$}
\DeclareUnicodeCharacter{D7}{$\times$}
\DeclareUnicodeCharacter{2200}{$\forall$}

\usepackage{xspace}

%%%%%%%%%%%%%%%%%%%%%%%%%%%%%%%%%%%%%%%%%%%%%%%%%%%%%%%%%%%%

\title{\textsc{Disco}: A Functional Programming Language for Discrete Mathematics}
\author{Brent A. Yorgey
\institute{Hendrix College\\ Conway, Arkansas, USA}
\email{yorgey@hendrix.edu}
}
\def\titlerunning{\textsc{Disco}: A Functional Programming Language for Discrete Mathematics}
\def\authorrunning{B. A. Yorgey}

%%%%%%%%%%%%%%%%%%%%%%%%%%%%%%%%%%%%%%%%%%%%%%%%%%%%%%%%%%%%

\newcommand{\disco}{\textsc{Disco}\xspace}

\begin{document}
\maketitle

XXX set up todo package

\begin{abstract}
  \disco is a pure functional programming language designed to be used
  in a Discrete Mathematics course.  XXX statically typed, math
  notation.
  features: property testing, arithmetic patterns, equirecursive
  types, subtyping.
\end{abstract}

\section{Introduction}
\label{sec:introduction}

Many computer science curricula at the university level include
\emph{discrete mathematics} as a core requirement \cite{ACM:2013}.
Often taken in the first or second year, a discrete mathematics course
introduces mathematical structures and techniques of foundational
importance in computer science, such as induction and recursion, set
theory, logic, modular arithmetic, functions, relations, and graphs.
In addition, it often serves as an introduction to writing formal
proofs.  Although there is wide agreement that discrete mathematics is
foundational, students sometimes struggle to see its relevance to
computer science.

\emph{Functional programming} is a style of programming, embodied in
languages such as Haskell, OCaml, Scala, F\#, and Racket, which
emphasizes functions (\emph{i.e.}\ input-output processes) rather than
sequences of instructions. It enables working at high levels of
abstraction as well as rapid prototyping and refactoring, and provides
a concise and powerful vocabulary to talk about many other topics in
computer science. It is seeing increasing adoption in industry, with
applications as diverse as high-frequency trading algorithms,
interactive web applications, and Facebook's anti-spam system.  For
these reasons, it is becoming critical to expose undergraduate
students to functional programming early, but many computer science
programs struggle to make space for it.  The Association for Computing
Machinery's 2013 curricular guidelines \cite{ACM:2013} do not include
functional programming as a core topic.

One creative idea is to combine functional programming and discrete
mathematics into a single course.  This is not a new idea
\cite{Wainwright:1992, Henderson:2002, Scharff:2002, Doets:2004,
  ODonnell:2006, VanDrunen:2011, Xing:2008}, and even shows up
in the 2007 model curriculum of the Liberal Arts Computer Science
Consortium \cite{LiberalArtsComputerScienceConsortium:2007}. The
benefits of such an approach are numerous:
\begin{itemize}
\item It allows functional programming to be introduced at an early
  point in undergraduates' careers, since discrete mathematics is
  typically taken in the first or second year.  This allows ideas from
  functional programming to inform students' thinking about the rest
  of the curriculum.  By contrast, when functional programming is left
  until later in the course of study, it is in danger of being seen as
  esoteric or as a mere curiosity.
\item The two subjects complement each other well: discrete math
  topics make good functional programming exercises, and ideas from
  functional programming help illuminate discrete math topics.
\item In a discrete mathematics course with both math and
  computer science majors, math majors can have a ``home turf
  advantage'' since the course deals with topics that may be already
  familiar to them (such as writing proofs), whereas computer science
  majors may struggle to connect the course content to computer
  science skills and concepts they already know.  Including functional
  programming levels the playing field, giving both groups of students
  a way to connect the course content to their previous experience.
  Computer science majors will be more comfortable learning math
  concepts that they can play with computationally; math majors can
  leverage their math experience to learn a bit about programming.
\item It is just plain fun: using programming enables interactive
  exploration of mathematics concepts, which leads to higher
  engagement and increased retention.
\end{itemize}

However, despite its benefits, this model is not widespread in
practice.  This may be due partly to lack of awareness, but there are
also some real roadblocks to adoption that make it impractical or
impossible for many departments.

\begin{itemize}
\item Existing functional languages---such as Haskell, Racket, OCaml,
  or SML---are general-purpose languages which (with the notable
  exception of Racket) were not designed specifically with teaching in
  mind.  The majority of their features are not needed in the setting
  of discrete mathematics, and teachers must waste a lot of time and
  energy explaining incidental detail or trying to hide it from
  students.
\item With the notable exception of Racket, tooling for existing
  functional languages is designed for professional programmers, not
  for students.  The systems can be difficult to set up, generate
  confusing error messages, and are generally designed to facilitate
  efficient production of code rather than interactive exploration and
  learning.
\item As with any subject, effective teaching of a functional language
  requires expertise in the language and its use, or at least thorough
  familiarity, on the part of the instructor. General-purpose
  functional languages are large, complex systems, requiring deep
  study and years of experience to master.  Even if only a small part
  of the language is presented to students, a high level of expertise
  is still required to be able to select and present a relevant subset
  of the language and to help students navigate around the features
  they do not need.  For many instructors, spending years learning a
  general-purpose functional language just to teach discrete
  mathematics is a non-starter.  This is especially a problem at
  schools where the discrete mathematics course is taught by
  mathematics faculty rather than computer science faculty.
\item There is often an impedance mismatch between standard
  mathematics notation and the notation used by existing functional
  programming languages.  As one simple example, in mathematics one
  can write $2x$ to denote multiplication of $x$ by $2$; but many
  programming languages require writing a multiplication operator, for
  example, \texttt{2*x}.  Any one such impedance mismatch is small, but taken
  as a whole they can be a real impediment to students as they move
  back and forth between the worlds of abstract mathematics and
  concrete computer programs.
\end{itemize}

\section{Disco by Example}

In order to introduce the main features of the language, XXX

\subsection{Greatest common divisor}
\label{sec:gcd}

Our first example is an implementation of the classic Euclidean
Algorithm for computing the greatest common divisor of two natural numbers.

XXX get these from files stored alongside the paper, make sure they check?
\begin{verbatim}
||| The greatest common divisor of two natural numbers.

!!! gcd(7,6)   == 1
!!! gcd(12,18) == 6
!!! gcd(0,0)   == 0
!!! forall a:N, b:N. gcd(a,b) divides a /\ gcd(a,b) divides b
!!! forall a:N, b:N, g:N. (g divides a /\ g divides b) ==> g divides gcd(a,b)

gcd : N * N -> N
gcd(a,0) = a                 -- base case
gcd(a,b) = gcd(b, a mod b)   -- recursive case
\end{verbatim}

Lines beginning with \texttt{|||} denote special documentation
comments attached to the subsequent definition (regular comments start with
\texttt{-{}-}).  This documentation can be later accessed with the
\texttt{:doc} command at the REPL prompt:

XXX automatically typeset REPL interactions from just input?
\begin{verbatim}
Disco> :doc gcd
Disco> :doc gcd
gcd : ℕ × ℕ → ℕ

The greatest common divisor of two natural numbers.
\end{verbatim}

Lines beginning with \texttt{!!!} denote \emph{tests} attached to the
subsequent definition, which can be either simple Boolean unit tests
(such as \verb|gcd(7,6) == 1|), or quantified properties (such as the
last two tests, which together express the universal property defining
\verb|gcd|).  Such properties will be tested exhaustively when
feasible, or, when exhaustive testing is impossible (as in this case),
tested with a finite number of randomly chosen inputs. (XXX using
QuickCheck + enumeration library).  For example:

\begin{verbatim}
Disco> :test forall a:N, b:N. let g = gcd(a,b) in g divides a /\ g divides b
  - Possibly true: ∀a, b. let g = gcd(a, b) in g divides a /\ g divides b
    Checked 100 possibilities without finding a counterexample.
Disco> :test forall a:N, b:N. let g = gcd(a,b) in g divides a /\ (2g) divides b
  - Certainly false: ∀a, b. let g = gcd(a, b) in g divides a /\ 2 * g divides b
    Counterexample:
      a = 0
      b = 1
\end{verbatim}

In the first case, Disco reports that 100 sample inputs were checked
without finding a counterexample, leading to the conclusion that the
property is \emph{possibly} true.  In the second case, when we modify
the test by demanding that \verb|b| must be divisible by twice
\verb|gcd(a,b)|, Disco is quickly able to find a counterexample,
proving that the property is \emph{certainly} false.

Every top-level definition in Disco must have a type signature;
\verb|gcd : N * N -> N| indicates that \verb|gcd| is a function which
takes a pair of natural numbers as input and produces a natural number
result.  The recursive definition of \verb|gcd| is then
straightforward, featuring multiple clauses and pattern-matching on
the input.

\subsection{Primality testing}
\label{sec:primetest}

The following example is taken from XXX Jan van Eijck, "The Haskell Road
to Logic, Maths, and Programming", 2nd Edition, pp. 4--11, and has
been transcribed from Haskell into \disco.

\begin{verbatim}
||| ldf k n calculates the least divisor of n that is at least k and
||| at most sqrt n.  If no such divisor exists, then it returns n.
ldf : N -> N -> N
ldf k n =
  {? k            if k divides n,
     n            if k^2 > n,
     ldf (k+1) n  otherwise
  ?}

||| ld n calculates the least nontrivial divisor of n, or returns n if
||| n has no nontrivial divisors.
ld : N -> N
ld = ldf 2

||| Test whether n is prime or not.
isPrime : N -> Bool
isPrime n = (n > 1) and (ld n == n)
\end{verbatim}

XXX Bool type, \verb|and| syntax.  Multiple syntaxes for many things.
Makes code harder to read perhaps but (1) get students used to the
idea of multiple ways to write things, (2) spend less time trying to
remember the precisely correct syntax; if they can remember one they
are good.  Case notation.

Support currying and partial application, but encourage tupling.

\subsection{Z-order}
\label{sec:zorder}

The ``Morton Z-order'' is one of my favorite bijections showing that
$\N \times \N$ has the same cardinality as $\N$; it takes a pair of
natural numbers, expresses them in binary, and interleaves their
binary representations to form a single natural number.

\begin{verbatim}
!!! forall n:N. zOrder(zOrder'(n)) == n
!!! forall p:N*N. zOrder'(zOrder(p)) == p

zOrder : ℕ×ℕ → ℕ
zOrder(0,0) = 0
zOrder(2m,n) = 2 * zOrder(n,m)
zOrder(2m+1,n) = 2 * zOrder(n,m) + 1

zOrder' : ℕ → ℕ×ℕ
zOrder'(0)    = (0,0)
zOrder'(2n)   = {? (2y,x) when zOrder'(n) is (x,y) ?}
zOrder'(2n+1) = {? (2y+1,x) when zOrder'(n) is (x,y) ?}
\end{verbatim}

\section{Functional Programming in Disco}
\label{sec:FP}

\disco is first and foremost a pure, functional programming language,
with all the usual features one might typically expect from such a thing:
\begin{itemize}
\item Functions are first-class objects
\item Anonymous functions can be written using lambda notation
\item Support for recursive algebraic data types and pattern-matching
\end{itemize}

Multi-argument functions can be expressed in curried form, although
the language and standard library are designed to encourage instead
taking tuples as arguments.  This is because tuple notation for
multi-argument functions is entirely standard in mathematics, and
students likely have already seen this notation in a previous
mathematics class.

\begin{verbatim}
f : N * N -> N
f(x,y) = x + 3y
\end{verbatim}

Of course, prior to taking a Discrete Math course, students probably
will not know about Cartesian product of sets, or how multi-argument
functions can be thought of as single-argument functions whose domain
is a product set/type; those topics are covered in the course.

\section{Mathematics in Disco}
\label{sec:math}

\disco supports the usual arithmetic operations (addition,
subtraction, multiplication, division, exponentiation, modulo,
rounding), comparisons, and Boolean operations.  Whenever possible,
\disco tries to support standard mathematical notation alongside more
traditional programming language notations.  For example:
\begin{itemize}
\item Multiplication may (often) be written via juxtaposition, without
  using an explicit multiplication operator:
\begin{verbatim}
Disco> x : N
Disco> x = 5
Disco> 3x + 7
22
Disco> (2 + 3)(5 - 1)
20
\end{verbatim}
\item In addition to traditional notation like \verb|&&| and
  \verb+||+ for binary operations, Disco also allows more
  math-inspired notation:
\begin{verbatim}
Disco> true /\ (false \/ (false -> true))
true
\end{verbatim}
\end{itemize}

\disco also has some operations which are less typical but
particularly useful in a Discrete Mathematics context, such as
factorial, binomial and multinomial coefficients, divisibility and
primality testing, and factoring.

\begin{verbatim}
Disco> 50!
30414093201713378043612608166064768844377641568960512000000000000
Disco> 10 choose 3
120
Disco> 10 choose [3,5,2]
2520
Disco> 13 divides 91
true
Disco> import num
Loading num.disco...
Disco> factor(2520)
⟅2 # 3, 3 # 2, 5, 7⟆
\end{verbatim}

XXX Lists, bags, and sets built in?

\disco allows ``arithmetic patterns'' in function definitions:
\begin{verbatim}
f : N -> N
f(3k) = k
f(3k+1) = 2k + 7
f(3k+2) = k^2
\end{verbatim}

Finally, ``case expressions'' XXX
\begin{verbatim}

\end{verbatim}

\section{Types}
\label{sec:types}

\section{Propositions, testing, and proving}
\label{sec:props}

\section{Tooling}
\label{sec:tools}

\section{Discussion}
\label{sec:discussion}

\cite{Yorgey:2012:promotion}

% The optional arguments of {\ttfamily $\backslash$documentclass$\{$eptcs$\}$} are
% \begin{itemize}
% \item at most one of
% {\ttfamily adraft},
% {\ttfamily submission} or
% {\ttfamily preliminary},
% \item at most one of {\ttfamily publicdomain} or {\ttfamily copyright},
% \item and optionally {\ttfamily creativecommons},
%   \begin{itemize}
%   \item possibly augmented with
%     \begin{itemize}
%     \item {\ttfamily noderivs}
%     \item or {\ttfamily sharealike},
%     \end{itemize}
%   \item and possibly augmented with {\ttfamily noncommercial}.
%   \end{itemize}
% \end{itemize}
% We use {\ttfamily adraft} rather than {\ttfamily draft} so as not to confuse hyperref.
% The style-file option {\ttfamily submission} is for papers that are
% submitted to {\ttfamily $\backslash$event}, where the value of the latter is
% to be filled in in line 2 of the tex file. Use {\ttfamily preliminary} only
% for papers that are accepted but not yet published. The final version
% of your paper to be uploaded to the EPTCS website should have
% none of these style-file options.

% Using the style-file option
% \href{http://creativecommons.org/licenses/}{creativecommons}
% authors equip their paper with a Creative Commons license that allows
% everyone to copy, distribute, display, and perform their copyrighted
% work and derivative works based upon it, but only if they give credit
% the way you request. By invoking the additional style-file option {\ttfamily
% noderivs} you let others copy, distribute, display, and perform only
% verbatim copies of your work, but not derivative works based upon
% it. Alternatively, the {\ttfamily sharealike} option allows others to
% distribute derivative works only under a license identical to the
% license that governs your work. Finally, you can invoke the option
% {\ttfamily noncommercial} that let others copy, distribute, display, and
% perform your work and derivative works based upon it for
% noncommercial purposes only.

% Authors' (multiple) affiliations and emails use the commands
% {\ttfamily $\backslash$institute} and {\ttfamily $\backslash$email}.
% Both are optional.
% Authors should moreover supply
% {\ttfamily $\backslash$titlerunning} and {\ttfamily $\backslash$authorrunning},
% and in case the copyrightholders are not the authors also
% {\ttfamily $\backslash$copyrightholders}.
% As illustrated above, heuristic solutions may be called for to share
% affiliations. Authors may apply their own creativity here \cite{multipleauthors}.

% EPTCS recommends using {\ttfamily $\backslash$documentclass[copyright,creativecommons]\{eptcs\}}.\\
% Additionally, the title should be set in \href{https://en.wikipedia.org/wiki/Title_case}{title case},
% meaning that major words start with a capital letter, and only articles, prepositions and
% conjunctions appear in lower case.

% \section{Ancillary files}

% Authors may upload ancillary files to be linked alongside their paper.
% These can, for instance, contain raw data for tables and plots in the
% article or program code.  Ancillary files are included with an EPTCS
% submission by placing them in a directory \texttt{anc} next to the
% main latex file. See also \url{https://arxiv.org/help/ancillary_files}.
% Please add a file README in the directory \texttt{anc}, explaining the
% nature of the ancillary files, as in
% \url{http://eptcs.org/paper.cgi?226.21}.

% \section{Prefaces}

% Volume editors may create prefaces using this very template,
% with {\ttfamily $\backslash$title$\{$Preface$\}$} and {\ttfamily $\backslash$author$\{\}$}.

\section{Bibliography}

% We request that you use
% \href{http://eptcs.web.cse.unsw.edu.au/eptcs.bst}
% {\ttfamily $\backslash$bibliographystyle$\{$eptcs$\}$}
% \cite{bibliographystylewebpage}, or one of its variants
% \href{http://eptcs.web.cse.unsw.edu.au/eptcsalpha.bst}{eptcsalpha},
% \href{http://eptcs.web.cse.unsw.edu.au/eptcsini.bst}{eptcsini} or
% \href{http://eptcs.web.cse.unsw.edu.au/eptcsalphaini.bst}{eptcsalphaini}
% \cite{bibliographystylewebpage}. Compared to the original {\LaTeX}
% {\ttfamily $\backslash$biblio\-graphystyle$\{$plain$\}$},
% it ignores the field {\ttfamily month}, and uses the extra
% bibtex fields {\ttfamily eid}, {\ttfamily doi}, {\ttfamily eprint} and {\ttfamily url}.
% The first is for electronic identifiers (typically the number $n$
% indicating the $n^\mathrm{th}$ paper in an issue) of papers in electronic
% journals that do not use page numbers. The other three are to refer,
% with life links, to electronic incarnations of the paper.

% \paragraph{DOIs}

% Almost all publishers use digital object identifiers (DOIs) as a
% persistent way to locate electronic publications. Prefixing the DOI of
% any paper with {\ttfamily https://doi.org/} yields a URI that resolves to the
% current location (URL) of the response page\footnote{Nowadays, papers
%   that are published electronically tend
%   to have a \emph{response page} that lists the title, authors and
%   abstract of the paper, and links to the actual manifestations of
%   the paper (e.g., as {\ttfamily dvi} or {\ttfamily pdf} file). Sometimes
%   publishers charge money to access the paper itself, but the response
%   page is always freely available.}
% of that paper. When the location of the response page changes (for
% instance, through a merge of publishers), the DOI of the paper remains
% the same and (through an update by the publisher) the corresponding
% URI will then resolve to the new location. For that reason, a reference
% ought to contain the DOI of a paper, with a live link to the corresponding
% URI, rather than a direct reference or link to the current URL of the
% publisher's response page. This is the r\^ole of the bibtex field {\ttfamily doi}.
% {\bfseries EPTCS requires the inclusion of a DOI in each cited paper, when available.}

% DOIs of papers can often be found through
% \url{http://www.crossref.org/guestquery};\footnote{For papers that will appear
%   in EPTCS and use \href{http://eptcs.web.cse.unsw.edu.au/eptcs.bst}
%   {\ttfamily $\backslash$bibliographystyle$\{$eptcs$\}$} there is no need to
%   find DOIs on this website, as EPTCS will look them up for you
%   automatically upon submission of the first version of your paper;
%   these DOIs can then be incorporated into the final version, together
%   with the remaining DOIs that need to be found at DBLP or the publisher's web pages.}
% the second method {\itshape Search on article title}, only using the {\bfseries
% surname} of the first-listed author, works best.
% Other places to find DOIs are DBLP and the response pages for cited
% papers (maintained by their publishers).

% \paragraph{The bibtex fields {\ttfamily eprint} and {\ttfamily url}}

% Often an official publication is only available against payment. However,
% as a courtesy to readers that do not wish to pay, the authors also
% make the paper available free of charge at a repository such as
% \url{arXiv.org}. In such a case, it is recommended to also refer and
% link to the URL of the response page of the paper in such a
% repository.  This can be done using the bibtex fields {\ttfamily eprint}
% or {\ttfamily url}.  The latter field should \textbf{not} be used
% to duplicate information that is also provided through {\ttfamily doi} or {\ttfamily eprint}.
% You can find archival-quality URLs for most recently published papers
% in DBLP, but please suppress repetition of DOI or {\ttfamily eprint} information though {\ttfamily url}.
% In fact, it is often useful to check your references against DBLP records anyway,
% or just find them there in the first place.

% \paragraph{Typesetting DOIs and URLs}

% When using {\LaTeX} rather than {\ttfamily pdflatex} to typeset your paper, by
% default no line breaks within long URLs are allowed. This leads often
% to very ugly output, that moreover is different from the output
% generated when using {\ttfamily pdflatex}. This problem is repaired when
% invoking \href{http://eptcs.web.cse.unsw.edu.au/breakurl.sty}
% {\ttfamily $\backslash$usepackage$\{$breakurl$\}$}: it allows line breaks
% within links and yield the same output as obtained by default with
% {\ttfamily pdflatex}.
% When invoking {\ttfamily pdflatex}, the package {\ttfamily breakurl} is ignored.

% The package {\ttfamily $\backslash$usepackage$\{$underscore$\}$} is
% recommended to deal with underscores in DOIs. This is not needed when
% using {\ttfamily $\backslash$usepackage$\{$breakurl$\}$} and not {\ttfamily pdflatex}.

% \paragraph{References to papers in the same EPTCS volume}

% To refer to another paper in the same volume as your own contribution,
% use a bibtex entry with
% \begin{center}
%   {\ttfamily series    = $\{\backslash$thisvolume$\{5\}\}$},
% \end{center}
% where 5 is the submission number of the paper you want to cite.
% You may need to contact the author, volume editors, or EPTCS staff to
% find that submission number; it becomes known (and unchangeable)
% as soon as the cited paper is first uploaded at EPTCS\@.
% Furthermore, omit the fields {\ttfamily publisher} and {\ttfamily volume}.
% Then in your main paper, you put something like:

% \noindent
% {\small \ttfamily $\backslash$providecommand$\{\backslash$thisvolume$\}$[1]$\{$this
%   volume of EPTCS, Open Publishing Association$\}$}

% \noindent
% This acts as a placeholder macro-expansion until EPTCS software adds
% something like

% \noindent
% {\small \ttfamily $\backslash$newcommand$\{\backslash$thisvolume$\}$[1]%
%   $\{\{\backslash$eptcs$\}$ 157$\backslash$opa, pp 45--56, doi:\dots$\}$},

% \noindent
% where the relevant numbers are pulled out of the database at publication time.
% Here the newcommand wins from the providecommand, and {\ttfamily \small $\backslash$eptcs}
% resp.\ {\ttfamily \small $\backslash$opa} expand to

% \noindent
% {\small \ttfamily $\backslash$sl Electronic Proceedings in Theoretical Computer Science} \hfill and\\
% {\small \ttfamily , Open Publishing Association} \hfill .

% \noindent
% Hence putting {\small \ttfamily $\backslash$def$\backslash$opa$\{\}$} in
% your paper suppresses the addition of a publisher upon expansion of the citation by EPTCS\@.
% An optional argument like
% \begin{center}
%   {\ttfamily series    = $\{\backslash$thisvolume[EPTCS]$\{5\}\}$},
% \end{center}
% overwrites the value of {\ttfamily \small $\backslash$eptcs}.

\bibliographystyle{eptcsalpha}
\bibliography{references}
\end{document}
