In 2007 the Liberal Arts Computer Science Consortium released a new
model of curriculum for computer science degrees at liberal arts
institutions \cite{LiberalArtsComputerScienceConsortium:2007}.  They
recommend that the best place to introduce functional programming is
in a discrete mathematics course.  They emphasize that introducing
students to multiple programming paradigms better prepares them for
the real world. It forces them to get used to thinking differently
which is a positive for all STEM students.

The ACM Special Interest Group on Programming Languages hosted a
workshop on undergraduate programming language curriculum
\cite{Allen:2008}.  Their results emphasized functional programming
due to the rising need to develop parallel algorithms, and the fact
that functional programming forces students to think differently.
They propose that functional programming be required for all computer
science students.

In 2011 VanDrunen argued that functional programming should be a core
part of a well-rounded computer science curriculum
\cite{VanDrunen:2011}.  He feared that based on computer science
curriculum design trends the functional programming paradigm would be
pushed out of undergraduate education. He proposes that the ideal
place to introduce functional programming is in a freshman or
sophomore level discrete mathematics course.  Just as we have argued
here VanDrunen makes the following points:
\begin{itemize}
\item ``Functional programming and discrete mathematics are closely
  related.''  For example, recursion and induction are fundamental to
  both,
\item ``Functional programming illuminates discrete mathematics.'' For
  example, the definition of what a function is when students come
  into discrete mathematics differs greatly from the definition they
  learn.  Functions as mathematical objects themselves can be a hard
  concept to grasp.  A functional programming languages gives an
  opportunity to let them play with the notion of higher-order
  functions to better their understanding.
\item ``Functional programming motivates discrete mathematics for
  computer science students.'' Many computer science students are not
  strong mathematics students despite the fact that computer science
  and mathematics are so closely related.  Functional programming has
  the ability to help these students connect with the mathematics by
  phrasing it in terms of what the most computer science students are
  excited about, namely, programming.
\end{itemize}
