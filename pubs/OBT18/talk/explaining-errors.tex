%% -*- mode: LaTeX; compile-command: "pdflatex explaining-errors.tex" -*-
\documentclass[xcolor=svgnames,12pt,aspectratio=169]{beamer}

%include polycode.fmt

\usepackage[all]{xy}
\usepackage{brent}
\usepackage{xspace}
\usepackage{fancyvrb}
\usepackage{ulem}
% \usepackage[backend=pgf,extension=pgf,input,outputdir=diagrams]{diagrams-latex}
\graphicspath{{images/}}

%%%%%%%%%%%%%%%%%%%%%%%%%%%%%%%%%%%%%%%%%%%%%%%%%%%%%%%%%%%%
%%%%%%%%%%%%%%%%%%%%%%%%%%%%%%%%%%%%%%%%%%%%%%%%%%%%%%%%%%%%

% Math typesetting

%% a bit more space for matrices
\setlength{\arraycolsep}{5pt}

\newcommand{\ty}[3]{{#1} \vdash {#2} : {#3}}
\newcommand{\nty}[3]{{#1} \nvdash {#2} : {#3}}

%%%%%%%%%%%%%%%%%%%%%%%%%%%%%%%%%%%%%%%%%%%%%%%%%%%%%%%%%%%%
%%%%%%%%%%%%%%%%%%%%%%%%%%%%%%%%%%%%%%%%%%%%%%%%%%%%%%%%%%%%

\newcommand{\etc}{\textit{etc.}}
\renewcommand{\eg}{\textit{e.g.}\xspace}
\renewcommand{\ie}{\textit{i.e.}\xspace}

\newcommand{\theschool}{Off the Beaten Track}
% \newcommand{\thelocation}{Los Angeles}
\newcommand{\thedate}{13 January 2018}

%%%%%%%%%%%%%%%%%%%%%%%%%%%%%%%%%%%%%%%%%%%%%%%%%%%%%%%%%%%%
%%%%%%%%%%%%%%%%%%%%%%%%%%%%%%%%%%%%%%%%%%%%%%%%%%%%%%%%%%%%

\setbeamertemplate{items}[circle]

\mode<presentation>
{
  \usetheme{default}                          % use a default (plain) theme

  \setbeamertemplate{navigation symbols}{}    % don't show navigation
                                              % buttons along the
                                              % bottom
  \setbeamerfont{normal text}{family=\sffamily}

  % XX remove this before giving actual talk!
  % \setbeamertemplate{footline}[frame number]
  % {%
  %   \begin{beamercolorbox}{section in head/foot}
  %     \vskip2pt
  %     \hfill \insertframenumber
  %     \vskip2pt
  %   \end{beamercolorbox}
  % }

  \AtBeginSection[]
  {
    \begin{frame}<beamer>
      \frametitle{}

      \begin{center}
%        \includegraphics[width=2in]{\sectionimg}
%        \bigskip

        {\Huge \insertsectionhead}
      \end{center}
    \end{frame}
  }
}

\defbeamertemplate*{title page}{customized}[1][]
{
  \vbox{}
  \vfill
  \begin{centering}
    \begin{beamercolorbox}[sep=8pt,center,#1]{title}
      \usebeamerfont{title}\inserttitle\par%
      \ifx\insertsubtitle\@@empty%
      \else%
        \vskip0.25em%
        {\usebeamerfont{subtitle}\usebeamercolor[fg]{subtitle}\insertsubtitle\par}%
      \fi%
    \end{beamercolorbox}%
    \vskip1em\par
    {\usebeamercolor[fg]{titlegraphic}\inserttitlegraphic\par}
    \vskip1em\par
    \begin{beamercolorbox}[sep=8pt,center,#1]{author}
      \usebeamerfont{author}\insertauthor
    \end{beamercolorbox}
    \begin{beamercolorbox}[sep=8pt,center,#1]{institute}
      \usebeamerfont{institute}\insertinstitute
    \end{beamercolorbox}
    \begin{beamercolorbox}[sep=8pt,center,#1]{date}
      \usebeamerfont{date}\insertdate
    \end{beamercolorbox}
  \end{centering}
  \vfill
}

\newenvironment{xframe}[1][]
  {\begin{frame}[fragile,environment=xframe,#1]}
  {\end{frame}}

% uncomment me to get 4 slides per page for printing
% \usepackage{pgfpages}
% \pgfpagesuselayout{4 on 1}[uspaper, border shrink=5mm]

% \setbeameroption{show only notes}

\renewcommand{\emph}{\textbf}

\title{Explaining Type Errors}
\date{\theschool \\ \thedate}
\author{\usebeamercolor[fg]{title}{Brent Yorgey} \and Richard
  Eisenberg \and Harley Eades}
% \titlegraphic{\includegraphics[width=1in]{deriv-tree}}

%%%%%%%%%%%%%%%%%%%%%%%%%%%%%%%%%%%%%%%%%%%%%%%%%%%%%%%%%%%%

\begin{document}

\begin{xframe}{}
   \titlepage
\end{xframe}

\begin{xframe}{The Dreaded Type Error Message}
\footnotesize
\begin{verbatim}
Could not deduce (Num t0)
from the context: (Num (t -> a), Num t, Num a)
  bound by the inferred type for ‘it’:
             forall a t. (Num (t -> a), Num t, Num a) => a
  at <interactive>:4:1-19
The type variable ‘t0’ is ambiguous
In the ambiguity check for the inferred type for ‘it’
To defer the ambiguity check to use sites, enable AllowAmbiguousTypes
When checking the inferred type
  it :: forall a t. (Num (t -> a), Num t, Num a) => a
\end{verbatim}
\end{xframe}

\begin{xframe}{}
  \begin{center}
    \includegraphics{what-the-function.jpg}
  \end{center}
\end{xframe}

\begin{xframe}{Theses}
  \begin{itemize}
  \item ``Improving'' error messages doesn't fundamentally help.
  \item Interactive \emph{error explanations} instead of static
    \emph{error messages}.
  \item \emph{Error explanation} = \emph{constructive evidence} for
    an error.
  \end{itemize}
\end{xframe}

\section{The Curse of Information}

\begin{xframe}{}
  \begin{center}
\begin{BVerbatim}
(\f -> f 3) (\p -> fst p)
\end{BVerbatim}
  \end{center}
\end{xframe}

\begin{xframe}{}
%  $\app{(\uabs{f}{\app{f}{3}})}{(\uabs{p}{\app{fst}{p}})}$

  \begin{center}
    \begin{BVerbatim}
(\f -> f 3) (\p -> fst p)
    \end{BVerbatim}

    \begin{verbatim}
Type mismatch between expected type (t, b0) and actual type Int
    \end{verbatim}
  \end{center}
\end{xframe}

\begin{xframe}
%  $\app{(\uabs{f}{\app{f}{3}})}{(\uabs{p}{\app{fst}{p}})}$

  \begin{center}
    \begin{BVerbatim}
(\f -> f 3) (\p -> fst p)
    \end{BVerbatim}

\begin{verbatim}
Type mismatch between expected type (t, b0) and actual type Int
  In the first argument of fst, namely p
  In the expression: fst p
  In the first argument of \ f -> f 3, namely
    (\ p -> fst p)
\end{verbatim}
  \end{center}
\end{xframe}

\begin{xframe}
  \footnotesize
  \begin{verbatim}
    Type mismatch between expected type (t, b0) and actual type Int
      In the first argument of fst, namely p
      In the expression: fst p
      In the first argument of \ f -> f 3, namely
        (\ p -> fst p)
      Relevant bindings include:
        fst :: (a,b) -> a
      Inferred types for subterms:
        3             :: Int
        (\f -> f 3)   :: forall a. (Int -> a) -> a
        (\p -> fst p) :: (Int -> a0)
        (\f -> f 3) (\p -> fst p) :: a0
      Suggested fixes:
        Change p to (p,y)
        Change fst to a function expecting an Int
        Change 3 to (x,y)
  \end{verbatim}
\end{xframe}

\begin{xframe}{}
  \begin{center}
    \includegraphics[width=3in]{not-helping.jpg}
  \end{center}
\end{xframe}

\begin{xframe}
  \footnotesize
  \begin{verbatim}
    Type mismatch between expected type (t, b0) and actual type Int
      In the first argument of fst, namely p
      In the expression: fst p
      In the first argument of \ f -> f 3, namely
        (\ p -> fst p)
      Relevant bindings include:
        fst :: (a,b) -> a
      Inferred types for subterms:
        3             :: Int
        (\f -> f 3)   :: forall a. (Int -> a) -> a
        (\p -> fst p) :: (Int -> a0)
        (\f -> f 3) (\p -> fst p) :: a0
      Suggested fixes:
        Change p to (p,y)
        Change fst to a function expecting an Int
        Change 3 to (x,y)
  \end{verbatim}
\end{xframe}

\begin{xframe}{}
  \Large
  \begin{center}
    \sout{MESSAGES} \\
    $\Downarrow$ \\
    EXPLANATIONS
  \end{center}
\end{xframe}

\begin{xframe}{}
  \small
\begin{verbatim}
p is expected to have a pair type but was inferred to have type Int.
  + Why is p expected to have a pair type?
  + Why was p inferred to have type Int?






\end{verbatim}
\end{xframe}

\begin{xframe}{}
  \small
\begin{verbatim}
p is expected to have a pair type but was inferred to have type Int.
  + Why is p expected to have a pair type?
  - Why was p inferred to have type Int?
    => p is the parameter of the lambda expression \p -> fst p, which
       must have type (Int -> a0).
    + Why must (\p -> fst p) have type (Int -> a0)?



\end{verbatim}
\end{xframe}

\begin{xframe}{}
  \small
\begin{verbatim}
p is expected to have a pair type but was inferred to have type Int.
  + Why is p expected to have a pair type?
  - Why was p inferred to have type Int?
    => p is the parameter of the lambda expression \p -> fst p, which
       must have type (Int -> a0).
    - Why must (\p -> fst p) have type (Int -> a0)?
      => It is an argument to (\f -> f 3), which was inferred to have
         type (∀a. (Int -> a) -> a).
      + Why was (\f -> f 3) inferred to have type (∀a. (Int -> a) -> a)?
\end{verbatim}
\end{xframe}

\begin{xframe}{Related work\dots}
  \begin{itemize}
  \item Plociniczak \& Odersky: Scalad (2012)
  \item Stuckey, Sulzmann \& Wazny: Chameleon (2003)
  \item Simon, Chitil, \& Huch (2000)
  \item Beaven \& Stansifer (1993)
  \end{itemize}

  Also\dots
  \begin{itemize}
  \item Seidel, Jhala \& Weimer: dynamic witnesses for errors
    (2016). A great idea, but not the focus of this talk.
  \end{itemize}
\end{xframe}
\section{Explaining errors}

\begin{xframe}{}
\begin{align*}
  t &::= x \mid \abs x \tau t \mid \app{t_1}{t_2} \\
  \tau &::= B \mid \tau_1 \to \tau_2 \\
  \Gamma &::= \cdot \mid \Gamma,x:\tau
\end{align*}
\end{xframe}

\begin{xframe}{}
\begin{mathpar}
  \inferrule{x : \tau \in \Gamma}{\ty \Gamma x \tau} \and
  \inferrule{\ty {\Gamma, x:\tau_1} t {\tau_2}}{\ty \Gamma {\abs x
      {\tau_1} t}{\tau_1 \to \tau_2}} \and
  \inferrule{\ty \Gamma {t_1} {\tau_1 \to \tau_2} \\ \ty \Gamma {t_2} {\tau_1}}
            {\ty \Gamma {\app{t_1}{t_2}} {\tau_2}}
\end{mathpar}
\end{xframe}

\begin{xframe}{}
  Type inference: Term -> Maybe Type

  Better: Term -> Maybe TypingDerivation

  Term -> Either UnTypingDerivation TypingDerivation

  see Ulf Norell,
  http://www.cse.chalmers.se/~ulfn/code/icfp2013/ICFP.html
\end{xframe}

\begin{xframe}
  Let's do a concrete example.

  Show rules for STLC, in Agda.  Show metatheorem.
\end{xframe}

\begin{xframe}
  Existing tools just let you explore typing derivations, e.g. to
  compare two of them that don't match, or with some place where they
  went wrong marked.

  Thinking in terms of untyping derivations gives us a lot more
  freedom to design explanations in whatever way we want.

  Options: untyping derivation is essentially a typing derivation with
  a hole.  Or with many holes.  But could be other things as well, or
  a combination.
\end{xframe}

\begin{xframe}
  Unification??
\end{xframe}

\end{document}